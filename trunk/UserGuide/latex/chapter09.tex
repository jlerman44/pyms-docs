% chapter09.tex

 %%%%%%%%%%%%%%%%%%%%%%%%%%%%%%%%%%%%%%%%%%%%%%%%%%%%%%%%%%%%%%%%%%%%%%%%%%%%%
 %                                                                           %
 %    PyMS documentation                                                     %
 %    Copyright (C) 2005-2010 Vladimir Likic                                 %
 %                                                                           %
 %    The files in this directory provided under the Creative Commons        %
 %    Attribution-NonCommercial-NoDerivs 2.1 Australia license               %
 %    http://creativecommons.org/licenses/by-nc-nd/2.1/au/                   %
 %    See the file license.txt                                               %
 %                                                                           %
 %%%%%%%%%%%%%%%%%%%%%%%%%%%%%%%%%%%%%%%%%%%%%%%%%%%%%%%%%%%%%%%%%%%%%%%%%%%%%

\chapter{Parallel processing with PyMS}

\section{\label{sec:mpi}Requirements}

Using PyMS parallel capabilities requires installation of the package
'mpi4py', which provides bindings of the Message Passing Interface (MPI)
for the Python programming language. This package can be downloaded
from {\tt http://code.google.com/p/mpi4py/}. Since 'mpi4py' provides
only Python bindings, it requires an MPI implementation. We recommend
using mpich2:\\
{\tt http://www.mcs.anl.gov/research/projects/mpich2/}\\
We show the installation of 'mpich2' and 'mpi2py' on Linux system from
software distributions downloaded from the projects' web site.

\subsection{\label{sec:mpich2}Installation of 'mpich2'}

\begin{enumerate}

\item From the mpich2 project web site download the current distribution of
mpich2 (in this case the file 'mpich2-1.2.1p1.tar.gz').

\end{enumerate}



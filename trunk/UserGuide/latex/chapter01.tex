% chapter01.tex

 %%%%%%%%%%%%%%%%%%%%%%%%%%%%%%%%%%%%%%%%%%%%%%%%%%%%%%%%%%%%%%%%%%%%%%%%%%%%%
 %                                                                           %
 %    PyMS documentation                                                     %
 %    Copyright (C) 2005-2010 Vladimir Likic                                 %
 %                                                                           %
 %    The files in this directory provided under the Creative Commons        %
 %    Attribution-NonCommercial-NoDerivs 2.1 Australia license               %
 %    http://creativecommons.org/licenses/by-nc-nd/2.1/au/                   %
 %    See the file license.txt                                               %
 %                                                                           %
 %%%%%%%%%%%%%%%%%%%%%%%%%%%%%%%%%%%%%%%%%%%%%%%%%%%%%%%%%%%%%%%%%%%%%%%%%%%%%

\chapter{Introduction}

PyMS is software for processing of chromatography--mass spectrometry data.
PyMS is written in Python programming language \cite{python}, and is
released as open source, under the GNU Public License version 2.

The driving idea behind PyMS is to provide a framework and a set of
components for rapid development and testing of methods for processing
of chromatography--mass spectrometry data. PyMS provides interactive
processing capabilities through any of the various interactive Python
front ends ("shells"). PyMS is essentially a Python library of
chromatography--mass spectrometry processing functions, and individual
commands can be collected into scripts which then can be run
non-interactively when it is preferable to run data processing in
the batch mode.

PyMS functionality consists of modules which are loaded when needed. 
For exlample, one such module provides display capabilities, and can
be used to display time dependent data (e.g. total ion chromatogram),
mass spectra, and signal peaks. This module is loaded only when
visualisation is needed. If one is interested only in noise smoothing,
only noise filtering functions are loaded into the Python environment
and used to smooth the data, while the visualisation module (and other
module) need not be loaded at all.

This modularity is supported on all levels of PyMS. For example, if
one is interested in noise filtering with the Savitsky-Golay filter,
only sub-module for Savitsky-Golay filter need to be loaded from
the noise smoothing module, disregarding other modules, as well
as other noise smoothing sub-modules. This organisation consisting
of hierarchical modules ensures that the processing pipeline is
put together from well defined modules each responsible for specific
functions; and furthermore different functionalities are completely
decoupled from one another, providing that implementing a new
functionality (such as test or prototype of a new algorithm) can
be implemented efficiently and ensuring that this will not break
any existing functionality.

\section{The PyMS project}

The PyMS project consists of three parts, and each of which exists
as a project in the Google Code repository that can be downloaded
separately. These three parts are:

\begin{itemize}
  \item pyms -- The PyMS code (http://code.google.com/p/pyms/)
  \item pyms-docs -- The PyMS documentation (http://code.google.com/p/pyms-docs/)
  \item pyms-test -- Examples of how to use PyMS (http://code.google.com/p/pyms-test/)
\end{itemize}

The project 'pyms' contains the source code of the Python package PyMS.
The project 'pyms-docs' contains PyMS style guide (relevant for those
who contribute to the PyMS code) and User Guide (this document). The
project 'pyms-test' contains tests and examples showing how to use
various PyMS features. These examples are explained in detail in 
subsequent chapters of this User Guide.

In addition, the current PyMS API documentation (releant for those
who are interested in PyMS internals) is available from here:\\
{\tt http://bioinformatics.bio21.unimelb.edu.au/pyms.api/index.html}\\

\section{PyMS installation}

PyMS is written in Python, and extensible and modular object-oriented
scripting language \cite{python}. Python is highly portable, cross-platform 
programming language which works well on all major modern operating
systems (Linux, MacOS X, Windows). PyMS is written in pure Python, and
therefore works on all platforms on which Python has been ported.

The PyMS is essentially a python library (a 'package' in python parlance,
which consists of several 'sub-packages'), and some of its functionality
depends on several Python libraries, such as 'numpy' (the Python library
for numerical computing), or 'matplotlib' (the Python library for plotting).
These also need to work on the operating system of your choice for the
functionality they provide to PyMS to be available. In general, the
libraries that PyMS uses are available for all operating systems. The
exception is 'pycdf' - a python interface to Unidata netCDF library
written by Andre Gosselin of the Institut Maurice-Lamontagne, Canada
(http://pysclint.sourceforge.net/pycdf/). {\em This library works only
under Linux/Unix and therefore PyMS functionality which depends on
it works only under the Unix operating system}.

There are several ways to install PyMS depending your computer
configuration and personal preferences. PyMS has been developed
on Linux, and a detailed installation instructions for Linux are
given below. PyMS should work on all major Linux distributions,
and has been tested extensively on Red Hat Linux.

\subsection{\label{subsec:code-download}Downloading PyMS source code}

PyMS source code resides on publicly accessible Google Code servers,
and can be accessed from the following URL: http://code.google.com/p/pyms/.
Under the section "Source" one can find the instructions for downloading
the source code. The same page provides the link under "This project's
Subversion repository can be viewed in your web browser" which allows
one to browse the source code on the server without actually downloading
it. Regardless of the target operating system, the first step towards
PyMS installation is to download the PyMS source code.

Google Code server maintains the source code with the program `subversion'
(an open-source version control system). To download the source code
one needs we use the subversions client. Several subversion clients are
available, some are open source, some freeware, and some are commerical
(for more information see http://subversion.tigris.org/). The svn client
programs are available for all operating systems. For example, on Linux
we use the svn client program which ships will most Linux systems called
simply 'svn'. The 'svn' exists for all mainstream operating
systems\footnote{For example, on Linux CentOS 4 it ships as a part of
the RPM package `subversion-1.3.2-1.rhel4.i386.rpm'}. A well known 
svn client for Windows is TortoiseSVN (http://tortoisesvn.tigris.org/).
TortoiseSVN provides graphical user interface, and is tightly integrated
with Windows. TortoiseSVN is open source and can be downloaded from the
project web page (http://tortoisesvn.tigris.org/). There are also several
commercial svn clients for Windowes.

Subversion has extensive functionality, however only the very basic
functionality is needed to download PyMS source code. For more information
about subversion please consult the book freely available
at http://svnbook.red-bean.com/.

If the computer is connected to the internet, and the subversion client
'svn' is installed. On Linux, the following command will download the
latest PyMS source code:

\begin{verbatim}
$ svn checkout http://pyms.googlecode.com/svn/trunk/ pyms
A    pyms/Peak
A    pyms/Peak/__init__.py
A    pyms/Peak/List
A    pyms/Peak/List/__init__.py
... [ further output deleted ] ...
Checked out revision 71.
\end{verbatim}

\section{\label{sec:external-libraries}External Libraries}

In addition to the source code, for the full PyMS functionality
several external libraries are required.

\subsection{\label{subsec:numpy}Package 'NumPy' (required for all
aspects of PyMS)}

The package NumPy is provides numerical capabilities to Python. This
package is used throughout PyMS (and also required for some external
packages used in PyMS), to its installation is mandatory.

The NumPy web site {\tt http://numpy.scipy.org/} provides the installation
instructions and the link to the source code.

\subsection{\label{subsec:pycdf}Package 'pycdf' (required for reading
ANDI-MS files)}

The pycdf (a python interface to Unidata netCDF library) source and
installation instructions can be downloaded from\\
{\tt http://pysclint.sourceforge.net/pycdf/}. Follow the installation
instructions to install pycdf.

\subsection{\label{subsec:pycluster}Package 'Pycluster' (required for peak
alignment by dynamic programming)}

The peak alignment by dynamic programming is located in the subpackage
pyms.Peak.List.DPA. This subpackage used the Python package `Pycluster'
as the clustering engine. Pycluster with its installation instructions
can be found here:\\
{\tt http://bonsai.ims.u-tokyo.ac.jp/~mdehoon/software/cluster/index.html}.

\subsection{\label{subsec:scipy-ndmage}Package 'scipy.ndimage' (required
for TopHat baseline corrector)}

If the full SciPy package is installed the 'ndimage' will be available.
However the SciPy contains extensive functionality, and its installation
is somewhat involved. Sometimes it is preferable to install only the
subpackage `ndimage'. This subpackage is provided as the PyMS-dependencies
gzipped file available for download from the PyMS webpage (see below).

\subsection{\label{subsec:matplotlib}Package 'matplotlib' (required 
for plotting)}

The displaying of information such as Ion Chromatograms and detected
peaks requires the package matplotlib. The matplotlib package can be
downloaded from:\\
{\tt http://matplotlib.sourceforge.net/}

\subsection{\label{subsec:mpi4py}Package 'mpi4py' (required 
for parallel processing)}

This package is required for parallel processing with PyMS. It can be
downloaded from:\\
{\tt http://code.google.com/p/mpi4py/}

\section{PyMS installation on Linux}

We recommend compiling your own Python installation before installing
PyMS. PyMS installation involves placing the PyMS code directory (pyms/)
into a location visible to the Python interpreter. This can be in the
standard place for 3rd party software (the directory site-packages/), or
alternatively if PyMS code is placed in a non-standard location the
Python interpreter needs to be made aware of it before before it is
possible to import PyMS modules. For more on this please consult the
Python command sys.path.append().

PyMS is currently being developed on Linux with the following packages:

\begin{verbatim}
Python-2.5.2
numpy-1.1.1
netcdf-4.0
pycdf-0.6-3b
ndimage.zip
Pycluster-1.41
matplotlib-0.99.1.2
mpi4py-1.2.1.tar.gz
mpich2-1.2.1p1.tar.gz
\end{verbatim}

For easy installation, we provide these packages bundled together into
the archive 'PyMS-Linux-deps-1.1.tar.gz' which can be downloaded from
the Bio21 Institute web server at the University of Melbourne:\\
http://bioinformatics.bio21.unimelb.edu.au/pyms.html\\

In addition to the dependencies bundle, one also needs to dowload the
PyMS source code as explained in the section \ref{subsec:code-download}).
Below we give a quick installation guide of packages required by PyMS
on Linux.

\begin{enumerate}

\item 'Python' installation:

\begin{verbatim}
$ tar xvfz Python-2.5.2.tgz
$ cd Python-2.5.2
$ ./configure
$ make
$ make install
\end{verbatim}

\noindent
This installs python in /usr/local/lib/python2.5. It is recommended
to make sure that python called from the command line is the one
just compiled and installed.

\item 'NumPy' installation:

\begin{verbatim}
$ tar xvfz numpy-1.1.1.tar.gz
$ cd numpy-1.1.1
$ python setup.py install
\end{verbatim}

\item 'pycdf' installation

Pycdf has two dependencies: the Unidata netcdf library and NumPy. The
NumPy installation is described above. To install pycdf, the netcdf
library must be downloaded\\
({\tt http://www.unidata.ucar.edu/software/netcdf/index.html}),\\
compiled and installed first:

\begin{verbatim}
$ tar xvfz netcdf.tar.gz
$ cd netcdf-4.0
$ ./configure
$ make
$ make install
\end{verbatim}

The last step will create several binary `libnetcdf*' files in
/usr/local/lib. Then 'pycdf' should be installed as follows:

\begin{verbatim}
$ tar xvfz pycdf-0.6-3b
$ cd pycdf-0.6-3b
$ python setup.py install
\end{verbatim}

\item 'Pycluster' installation

\begin{verbatim}
$ tar xvfz Pycluster-1.42.tar.gz
$ cd Pycluster-1.42
$ python setup.py install
\end{verbatim}

\item 'ndimage' installation:

\begin{verbatim}
$ unzip ndimage.zip
$ cd ndimage
$ python setup.py install --prefix=/usr/local
\end{verbatim}

\noindent
Since 'ndimage' was installed outside the scipy package, this requires
some manual tweaking:

\begin{verbatim}
$ cd /usr/local/lib/python2.5/site-packages
$ mkdir scipy
$ touch scipy/__init__.py
$ mv ndimage scipy
\end{verbatim}

\item 'matplotlib' installation:

\begin{verbatim}
$ tar xvfz matplotlib-0.99.1.2
$ cd matplotlib-0.99.1.1
$ python setup.py build
$ python setup.py install
\end{verbatim}

\noindent
The 'pyms.Display' module uses the TKAgg backend for matplotlib. If
this is not your default backend, the matplotlibrc file may be edited.
To locate the file 'matplotlibrc', in a python interactive session:

\begin{verbatim}
>>> import matplotlib
>>> matplotlib.matplotlib_fname()
\end{verbatim}

\noindent
Open the matplotlibrc file in a text editor and adjust the 'backend'
parameter to 'TKAgg'.

\item 'mpi4py' installation:

This package is required for running PyMS processing on multiple processors
(CPUs). Instructions how to install this package and run PyMS processing
in parallel are given in Section \ref{sec:mpi}. 


\end{enumerate}


\section{Troubleshooting}

The most likely problem with PyMS installation is a problem with
installing one of the PyMS dependencies.

\subsection{Pycdf import error}

On Red Hat Linux 5 the SELinux is enabled by default, and this causes the
following error while trying to import properly installed pycdf:

\begin{verbatim}
$ python
Python 2.5.2 (r252:60911, Nov  5 2008, 16:25:39)
[GCC 4.1.1 20070105 (Red Hat 4.1.1-52)] on linux2
Type "help", "copyright", "credits" or "license" for more information.
>>> import pycdf
Traceback (most recent call last):
  File "<stdin>", line 1, in <module>
  File "/usr/local/lib/python2.5/site-packages/pycdf/__init__.py",
        line 22, in <module> from pycdf import *
  File "/usr/local/lib/python2.5/site-packages/pycdf/pycdf.py",
        line 1096, in <module> import pycdfext as _C
  File "/usr/local/lib/python2.5/site-packages/pycdf/pycdfext.py",
        line 5, in <module> import _pycdfext
ImportError: /usr/local/lib/python2.5/site-packages/pycdf/_pycdfext.so:
    cannot restore segment prot after reloc: Permission denied
\end{verbatim}

This problem is removed simply by disabling SELinux (login as `root',
open the menu Administration $\rightarrow$ Security Level and Firewall,
tab SELinux, change settings from `Enforcing' to `Disabled').

This problem is likely to occur on Red Hat Linux derivative distributions
such as CentOS.

\section{PyMS tutorial and examples}

This document provides extensive tutorial on the use of PyMS, and the
corresponding examples can be downloaded from the publicly accessible
Google code project 'pyms-test' (http://code.google.com/p/pyms-test/). 
The data used in PyMS documentation and examples is available from the
Bio21 Institute server at the University of Melbourne. Please follow
the link from the PyMS web page:\\
{\tt http://bioinformatics.bio21.unimelb.edu.au/pyms}\\
or go directly to\\
{\tt http://bioinformatics.bio21.unimelb.edu.au/pyms-data/}\\

A tutorial illustrating various PyMS features in detail is provided
in subsequent chapters of this User Guide. The commands executed
interactively are grouped together by example, and provided as
Python scripts in the project 'pyms-test' (this is a Google code
project, similar to the project 'pyms' which contains the PyMS
source code).

The setup used in the examples below is as follows. The projects 'pyms',
'pyms-test', 'pyms-docs', and 'data' are all in the same directory,
'/x/PyMS'. In the project 'pyms-test' there is a directory corresponding
to each example, coded with the chapter number (ie. {\tt pyms-test/21a/}
corresponds to the Example 21a, from Chapter 2).

In each example directory, there is a script named 'proc.py' which
contains the commands given in the example.  Provided that the paths
to 'pyms' and 'pyms-data' are set properly, these scripts could be
run with the following command:

\begin{verbatim}
$ python proc.py
\end{verbatim}

Before running each example the Python interpreter was made aware of
the PyMS location with the following commands:

\begin{verbatim}
import sys
sys.path.append("/x/PyMS")
\end{verbatim}

For brevity these commands will not be shown in the examples below,
but they are included in 'pyms-test' example scripts.  The above
path may need to be adjusted to match your own directory structure.

\section{\label{sec:pyms-windows}Using PyMS on Windows}

Python is highly cross-platform compatible, and PyMS works seamlessly
on Windows. The only exception is reading of data in ANDI-MS (NetCDF)
format, widely used standard format for storing raw
chromatography--mass spectrometry data (see \ref{sec:chapter2-introduction})
This capability in PyMS depends on the 'pycdf' library, which is not
supported on Windows (see the Subsection \ref{subsec:pycdf}). Therefore
at present the ability to read ANDI-MS files is limited to Linux. All
other PyMS functionality is available under Windows.

We use Linux for the development and deployment PyMS, and this User
Guide largely assumes that PyMS is used under Linux. In this Subsection
we give some pointers on how to use PyMS under Windows.

\subsection{PyMS installation on Windows}

\begin{enumerate}

\item Install Python, NumPy, SciPy, and matplotlib for Windows.
The bundle of these software packages tested under Windows XP and
Windows 7 can be downloaded from the PyMS project page at the
Bio21 Institute, University of Melbourne\\
{\tt http://bioinformatics.bio21.unimelb.edu.au/pyms.html}

\item Download the latest PyMS code from the PyMS Google Code
project page\\
{\tt http://code.google.com/p/pyms/}

\item Unpack the PyMS code and place it in a standard place for
Python libraries, or adjust the PYTHONPATH variable to include
the path to PyMS. If Python is installed in C:\\Python25, the
standard place for Python libraries is C:\\Python25\\Libs\\site-packages

\item Start IDLE, or other Python shell. If PyMS is installed
properly, the following command will not return any warning or
error messages:

\begin{verbatim}
>>> import pyms
\end{verbatim}

\end{enumerate}

\subsection{Example processing GC-MS data with PyMS on Windows}

The example data can be downloaded from\\
{\tt http://bioinformatics.bio21.unimelb.edu.au/pyms/data/}.
We will use the data file in the JCAMP-DX format, named
{\tt gc01\_0812\_066.jdx}. Once downloaded this data file
will be placed in the folder {\tt C:\\Data}.

The Python environment can be accessed from several Python
shells. The default shell that comes with the Python 2.5
installation is IDLE. In this example we first load the
raw data,

\begin{verbatim}
>>> from pyms.GCMS.IO.JCAMP.Function import JCAMP_reader
>>> jcamp_file = "C:\Data\gc01_0812_066.jdx"
>>> data = JCAMP_reader(jcamp_file)
 -> Reading JCAMP file 'C:\Data\gc01_0812_066.jdx'
\end{verbatim}

Then build the intensity matrix object by binning:

\begin{verbatim}
>>> from pyms.GCMS.Function import build_intensity_matrix_i
>>> im = build_intensity_matrix_i(data)
\end{verbatim}

We obtain the dimensions of the intensity matrix, then loop
over all ion chromatograms, and for each ion chromatogram
apply Savitzky-Golay noise filter and tophat baseline correction.
The resulting noise and baseline corrected ion chromatogram
is saved back into the intensity matrix:

\begin{verbatim}
>>> n_scan, n_mz = im.get_size()
>>> from pyms.Noise.SavitzkyGolay import savitzky_golay
>>> from pyms.Baseline.TopHat import tophat
>>> for ii in range(n_mz):
	print "working on IC", ii
	ic = im.get_ic_at_index(ii)
	ic1 = savitzky_golay(ic)
	ic_smooth = savitzky_golay(ic1)
	ic_base = tophat(ic_smooth, struct="1.5m")
	im.set_ic_at_index(ii, ic_base)
\end{verbatim}



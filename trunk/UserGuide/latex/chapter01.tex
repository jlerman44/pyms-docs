% chapter01.tex

 %%%%%%%%%%%%%%%%%%%%%%%%%%%%%%%%%%%%%%%%%%%%%%%%%%%%%%%%%%%%%%%%%%%%%%%%%%%%%
 %                                                                           %
 %    PyMS documentation                                                     %
 %    Copyright (C) 2005-2010 Vladimir Likic                                    %
 %                                                                           %
 %    The files in this directory provided under the Creative Commons        %
 %    Attribution-NonCommercial-NoDerivs 2.1 Australia license               %
 %    http://creativecommons.org/licenses/by-nc-nd/2.1/au/                   %
 %    See the file license.txt                                               %
 %                                                                           %
 %%%%%%%%%%%%%%%%%%%%%%%%%%%%%%%%%%%%%%%%%%%%%%%%%%%%%%%%%%%%%%%%%%%%%%%%%%%%%

\chapter{Introduction}

PyMS is a Python package and consists sub-packages written in Python
programming language \cite{python}. PyMS is released as open source,
under the GNU Public License version 2.

The main idea behind PyMS is to provide a framework and a set of
components for rapid development and testing of methods for processing
of chromatography--mass spectrometry data. PyMS provides interactive
processing capabilities through any of the various interactive Python
front ends ("shells"). PyMS is essentially a Python library of
chromatography--mass spectrometry processing functions, and individual
commands can be collected into scripts which then can be run
non-interactively when it is preferable to run data processing in
the batch mode.

PyMS functionality consists of modules which are loaded when needed. 
For exlample, one such module provides display capabilities, and can
be used to display time dependent data (e.g. total ion chromatogram),
mass spectra, and signal peaks. This module is loaded only when
visualisation is needed. If one is interested only in noise smoothing,
only noise filtering functions are loaded into the Python environment
and used to smooth the data, while the visualisation module (and other
module) need not be loaded at all.

This modularity is supported on all levels of PyMS. For example, if
one is interested in noise filtering with the Savitsky-Golay filter,
only sub-module for Savitsky-Golay filter need to be loaded from
the noise smoothing module, disregarding other modules, as well
as other noise smoothing sub-modules. This organisation consisting
of hierarchical modules ensures that the processing pipeline is
put together from well defined modules each responsible for specific
functions; and furthermore different functionalities are completely
decoupled from one another, providing that implementing a new
functionality (such as test or prototype of a new algorithm) can
be implemented efficiently and ensuring that this will not break
any existing functionality.

\section{The PyMS project}

There are three parts of the pyms project, each existing as a separate
project in the Google Code repository that can be downloaded separately.
These are:

\begin{itemize}
  \item pyms -- The PyMS code
  \item pyms-docs -- The PyMS documentation
  \item pyms-test -- Examples of how to use PyMS
\end{itemize}

The data used in PyMS documentation and examples is available from the
Bio21 Institute server at the University of Melbourne:\\
{\tt http://bioinformatics.bio21.unimelb.edu.au/pyms-data/}\\
In addition, the current PyMS API documentation is available from here:\\
{\tt http://bioinformatics.bio21.unimelb.edu.au/pyms.api/index.html}

\section{PyMS installation on Linux}

There are several ways to install PyMS depending your computer configuration
and preferences. The recommended way install PyMS is to compile Python
from sources and install PyMS within the local Python installation. This
procedure is described below.

PyMS has been developed on Linux, and a detailed installation instructions
for Linux are given below. PyMS has been tested on several major Linux
distributions, including Red Hat Linux.

\subsection{Downloading PyMS source code}

PyMS source code resides on Google Code servers, and can be accessed
from the following URL: http://code.google.com/p/pyms/. Under the
section "Source" one can find the instructions for downloading the
source code. The same page provides the link under "This project's
Subversion repository can be viewed in your web browser" which allows
one to browse the source code on the server without actually downloading
it.

Google Code maintains the source code with the program `subversion'
(an open-source version control system). To download the source code
one needs we use the subversions client program called `svn'. The `svn'
client exists for all mainstream operating systems\footnote{For example,
on Linux CentOS 4 we have installed the RPM package
`subversion-1.3.2-1.rhel4.i386.rpm' to provide us with the subversion
client `svn'.}, for more information see http://subversion.tigris.org/.
The book about subversion is freely available on-line at
http://svnbook.red-bean.com/. Subversion has extensive functionality,
however only the very basic functionality is needed to download PyMS
source code.

If the computer is connected to the internet and the subversion client
is installed, the following command will download the latest PyMS source
code:

\begin{verbatim}
$ svn checkout http://pyms.googlecode.com/svn/trunk/ pyms
A    pyms/Peak
A    pyms/Peak/__init__.py
A    pyms/Peak/List
A    pyms/Peak/List/__init__.py
.....
Checked out revision 71.
\end{verbatim}

\subsection{PyMS installation}

PyMS installation involves placing the PyMS code directory (pyms/) into
a location visible to the Python interpreter. This can be in the standard
place for 3rd party software (the directory site-packages/). If PyMS code
is placed in a non-standard place the Python interpreter needs to be made
aware of it before before it is possible to import PyMS modules (see the
Python sys.path.append() command).

We recommend compiling your own Python installation for PyMS.

In addition to the PyMS core source code, a number of external packages
is used to provide additional functionality. These are explained below.

\subsection{\label{subsec:numpy}Package `NumPy'}

The package NumPy is provides numerical capabilities to Python. This
package is used throughout PyMS (and also required for some external
packages used in PyMS), to its installation is mandatory.

The NumPy web site {\tt http://numpy.scipy.org/} provides the installation
instructions and the link to the source code.

\subsection{\label{subsec:pycdf}Package `pycdf' (required for reading
ANDI-MS files)}

The pycdf (a python interface to Unidata netCDF library) source and
installation instructions can be downloaded from\\
{\tt http://pysclint.sourceforge.net/pycdf/}. Follow the installation
instructions to install pycdf.

\subsection{\label{subsec:pycluster}Package `Pycluster' (required for peak
alignment by dynamic programming)}

The peak alignment by dynamic programming is located in the subpackage
pyms.Peak.List.DPA. This subpackage used the Python package `Pycluster'
as the clustering engine. Pycluster with its installation instructions
can be found here:\\
{\tt http://bonsai.ims.u-tokyo.ac.jp/~mdehoon/software/cluster/index.html}.

\subsection{\label{subsec:scipy-ndmage}Package `scipy.ndimage' (required
for TopHat baseline corrector)}

If the full SciPy package is installed the `ndimage' will be available. However
the SciPy contains large amount of functionality, and its installation is
somewhat involved. In some situations in may be preferable to install only
the subpackage `ndimage'. This subpackage is provided in the PyMS-dependencies gzipped
file available for download from the PyMS webpage.

\subsection{\label{subsec:matplotlib}Package 'matplotlib' (required 
for plotting information in Display module)}

The displaying of information such as Ion Chromatograms and detected peaks
is done using the package matplotlib. The matplotlib package and user information
can be accessed at:\\
{\tt http://matplotlib.sourceforge.net/}

\section{Current PyMS development environment}

PyMS is currently being developed with the following packages:

\begin{verbatim}
Python-2.5.2
numpy-1.1.1
netcdf-4.0
pycdf-0.6-3b
Pycluster-1.41
matplotlib-0.99.1.2
\end{verbatim}

A quick installation guide for packages required by PyMS is given below.

\begin{enumerate}

\item Python installation:

\begin{verbatim}
$ tar xvfz Python-2.5.2.tgz
$ cd Python-2.5.2
$ ./configure
$ make
$ make install
\end{verbatim}

\noindent
This installs python in /usr/local/lib/python2.5.  Make sure that python called
from the command line is the one just compiled and installed.

\item NumPy installation:

\begin{verbatim}
$ tar xvfz numpy-1.1.1.tar.gz
$ cd numpy-1.1.1
$ python setup.py install
\end{verbatim}

\item pycdf installation

Pycdf has two dependencies: the Unidata netcdf library and NumPy. The NumPy
installation is described above. To install pycdf, the netcdf library must
be downloaded\\
({\tt http://www.unidata.ucar.edu/software/netcdf/index.html}),\\
compiled and installed first:

\begin{verbatim}
$ tar xvfz netcdf.tar.gz
$ cd netcdf-4.0
$ ./configure
$ make
$ make install
\end{verbatim}

The last step will create several binary `libnetcdf*' files in /usr/local/lib.
pycdf can be installed as follows:

\begin{verbatim}
$ tar xvfz pycdf-0.6-3b
$ cd pycdf-0.6-3b
$ python setup.py install
\end{verbatim}

\item Pycluster installation

\begin{verbatim}
$ tar xvfz Pycluster-1.42.tar.gz
$ cd Pycluster-1.42
$ python setup.py install
\end{verbatim}

\item ndimage installation:

\begin{verbatim}
$ unzip ndimage.zip
$ cd ndimage
$ python setup.py install --prefix=/usr/local
\end{verbatim}

\noindent
Since ndimage was installed outside the scipy package, this requires some manual
correction:

\begin{verbatim}
$ cd /usr/local/lib/python2.5/site-packages
$ mkdir scipy
$ touch scipy/__init__.py
$ mv ndimage scipy
\end{verbatim}

\item matplotlib installation:

\begin{verbatim}
$ tar xvfz matplotlib-0.99.1.2
$ cd matplotlib-0.99.1.1
$ python setup.py build
$ python setup.py install
\end{verbatim}

\noindent
The pyms.Display package uses the TKAgg backend for matplotlib. As this is
not the default backend, the matplotlibrc file must be edited. To locate the
matplotlibrc file, in a python interactive session:

\begin{verbatim}
>>> import matplotlib
>>> matplotlib.matplotlib_fname()
\end{verbatim}

\noindent
Open the matplotlibrc file in a text editor and adjust the 'backend'
parameter to 'TKAgg'.

\end{enumerate}

\section{Troubleshooting}

The PyMS is essentially a python library (a `package' in python parlance, which
consists of several `sub-packages'), which for some functionality depends on
other python libraries, such as NumPy, pycdf, and Pycluster. The most likely
problem with PyMS installation is a problem with installing one of the PyMS
dependencies.

\subsection{Pycdf import error}

On Red Hat Linux 5 the SELinux is enabled by default, and this causes the
following error while trying to import properly installed pycdf:

\begin{verbatim}
$ python
Python 2.5.2 (r252:60911, Nov  5 2008, 16:25:39)
[GCC 4.1.1 20070105 (Red Hat 4.1.1-52)] on linux2
Type "help", "copyright", "credits" or "license" for more information.
>>> import pycdf
Traceback (most recent call last):
  File "<stdin>", line 1, in <module>
  File "/usr/local/lib/python2.5/site-packages/pycdf/__init__.py", line 22, in <module>
    from pycdf import *
  File "/usr/local/lib/python2.5/site-packages/pycdf/pycdf.py", line 1096, in <module>
    import pycdfext as _C
  File "/usr/local/lib/python2.5/site-packages/pycdf/pycdfext.py", line 5, in <module>
    import _pycdfext
ImportError: /usr/local/lib/python2.5/site-packages/pycdf/_pycdfext.so:
    cannot restore segment prot after reloc: Permission denied
\end{verbatim}

This problem is removed simply by disabling SELinux (login as `root', open the
menu Administration $\rightarrow$ Security Level and Firewall, tab SELinux,
change settings from `Enforcing' to `Disabled').

This problem is likely to occur on Red Hat Linux derivative distributions such
as CentOS.

\section{PyMS tutorial and examples}

A tutorial illustrating various PyMS features in provided in subsequent chapter
of this User Guide. The commands executed interactively are grouped together
by example, and provided as Python scripts in the project `pyms-test' (this is
a Google code project, similar to the project `pyms' which contains the PyMS
source code).

The setup used in the examples below is as follows. The projects `pyms',
`pyms-test', `pyms-docs', and `data' are all in the same directory,
`/x/PyMS'. In the project `pyms-test' there is a directory corresponding to
each example coded with the example number (ie. {\tt pyms-test/21a/}
corresponds to Example 1a in Chapter 2).

In each example directory, there is a script named `proc.py' which contains
the commands given in the example.  Provided that the paths to `pyms' and
`pyms-data' are set properly, these scripts could be run with the following
command:

\begin{verbatim}
$ python proc.py
\end{verbatim}

Before running each example the Python interpreter was made aware of the
PyMS location with the following commands:

\begin{verbatim}
import sys
sys.path.append("/x/PyMS")
\end{verbatim}

For brevity these commands will not be shown in the examples below, but
they are included in `pyms-test' example scripts.  The above path may need
to be adjusted to match your own directory structure.

All data files (raw data files, peak lists etc.) used in the example below
can be found at \\
{\tt http://bioinformatics.bio21.unimelb.edu.au/pyms/data/} \\
and are assumed to be located in the `data' directory.



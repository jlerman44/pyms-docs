% chapter05.tex

 %%%%%%%%%%%%%%%%%%%%%%%%%%%%%%%%%%%%%%%%%%%%%%%%%%%%%%%%%%%%%%%%%%%%%%%%%%%%%
 %                                                                           %
 %    PyMS documentation                                                     %
 %    Copyright (C) 2005-2010 Vladimir Likic                                 %
 %                                                                           %
 %    The files in this directory provided under the Creative Commons        %
 %    Attribution-NonCommercial-NoDerivs 2.1 Australia license               %
 %    http://creativecommons.org/licenses/by-nc-nd/2.1/au/                   %
 %    See the file license.txt                                               %
 %                                                                           %
 %%%%%%%%%%%%%%%%%%%%%%%%%%%%%%%%%%%%%%%%%%%%%%%%%%%%%%%%%%%%%%%%%%%%%%%%%%%%%

\chapter{Peak detection and representation}

\section{Peak Object}

Fundamental to GC-MS analysis is the identification of individual components
of the sample mix. The basic component unit is represented as a signal peak.
In PyMS a signal peak is represented as `Peak' object (the class defined in
{\tt pyms.Peak.Class}). PyMS provides functions to detect peaks and create
(discussed at the end of the chapter).

A peak object stores a minimal set of information about a signal peak, namely,
the retention time at which the peak apex occurs and the mass spectra at the
apex. Additional information, such as, peak width, TIC and individual ion areas
can be filtered from the GC-MS data and added to the Peak object information.

\subsection{Creating a Peak Object}

\noindent
[ {\em This example is in pyms-test/50} ]

A peak object can be created for a scan at a given retention time by providing
the retention time (in minutes or seconds) and the MassSpectrum object of the
scan. In the example below, first a file is loaded and an IntensityMatrix,
{\em im}, built, then a MassSpectrum, {\em ms}, can be selected at a given
time (31.17 minutes in this example).

\begin{verbatim}
>>> from pyms.GCMS.Function import build_intensity_matrix_i
>>> from pyms.GCMS.IO.ANDI.Function import ANDI_reader
>>> andi_file = "/x/PyMS/data/gc01_0812_066.cdf"
>>> data = ANDI_reader(andi_file)
>>> im = build_intensity_matrix_i(data)
>>> index = im.get_index_at_time(31.17*60.0)
>>> ms = im.get_ms_at_index(index)
\end{verbatim}

\noindent
Now a Peak object can be created for the given retention time and MassSpectrum.

\begin{verbatim}
>>> from pyms.Peak.Class import Peak
>>> peak = Peak(31.17, ms, minutes=True)
\end{verbatim}

\noindent
By default the retention time is assumed to be in seconds. The parameter
{\tt minutes} can be set to True if the retention time is given in minutes. As a
matter of convention, PyMS internally stores retention times in seconds, so the
{\tt minutes} parameter ensures the input and output of the retention time are
in the same units.

\subsection{Peak Object properties}

\noindent
[ {\em This example is in pyms-test/50} ]

The retention time of the peak can be returned with {\tt get\_rt()}. The
retention time is returned in seconds with this method. The mass spectrum
can be returned with {\tt get\_mass\_spectrum()}.

The Peak object constructs a unique identification (UID) based on the spectrum
and retention time. This helps in managing lists of peaks (covered in the next
chapter). The UID can be returned with {\tt get\_UID()}. The format of the UID
is the masses of the two most abundant ions in the spectrum, the ratio of the
abundances of the two ions, and the retention time (in the same units as given
when the Peak object was created). The format is {\tt Mass1-Mass2-Ratio-RT}. For
example,

\begin{verbatim}
>>> print peak.get_rt()
1870.2
>>> print peak.get_UID()
319-73-74-31.17
\end{verbatim}

\subsection{Modifying a Peak Object}

\noindent
[ {\em This example is in pyms-test/51} ]

The Peak object has methods for modifying the mass spectrum. The mass range can
be cropped to a smaller range with {\tt crop\_mass()}, and the intensity
values for a single ion can be set to zero with {\tt null\_mass()}. For
example, the mass range can be set from 60 to 450 m/z, and the ions related to
sample preparation can be ignored by setting their intensities to zero;

\begin{verbatim}
>>> peak.crop_mass(60, 450)
>>> peak.null_mass(73)
>>> peak.null_mass(147)
\end{verbatim}

\noindent
The UID is automatically updated to reflect the changes;

\begin{verbatim}
>>> print peak.get_UID()
319-205-54-31.17
\end{verbatim}

It is also possible to change the peak mass spectrum by calling the
method {\tt set\_mass\_spectrum()}.

\section{Peak detection}
\label{sec:peak-detection}

The general use of a Peak object is to extract them from the GC-MS data and
build a list of peaks. In PyMS, the function for peak detection is based on the
method of Biller and Biemann (1974)\cite{biller74}. The basic process is to find
all maximising ions in a pre-set window of scans, for a given scan.  The ions
that maximise at a given scan are taken to belong to the same peak.

The function is {\tt BillerBiemann()} in
{\tt pyms.Deconvolution.BillerBiemann.Function}. The function has parameters for
the window width for detecting the local maxima ({\tt points}), and the number
of {\tt scans} across which neighbouring, apexing, ions are combined and
considered as belonging to the same peak. The number of neighbouring scans to
combine is related to the likelyhood of detecting a peak apex at a single scan
or several neighbouring scans. This is more likely when there are many scans
across the peak. It is also possible, however, when there are very few scans
across the peak. The scans are combined by taking all apexing ions to have
occurred at the scan that had to greatest TIC prior to combining scans.

\subsection{Sample processing and Peak detection}

\noindent
[ {\em This example is in pyms-test/52} ]

The process for detecting peaks is to pre-process the data by performing noise
smoothing and baseline correction on each ion (as in {\em pyms-test/52}). The
first steps then are;

\begin{verbatim}
>>> from pyms.GCMS.IO.ANDI.Function import ANDI_reader
>>> from pyms.GCMS.Function import build_intensity_matrix
>>> from pyms.Noise.SavitzkyGolay import savitzky_golay
>>> from pyms.Baseline.TopHat import tophat
>>>
>>> andi_file = "/x/PyMS/data/gc01_0812_066.cdf"
>>> data = ANDI_reader(andi_file)
>>>
>>> im = build_intensity_matrix(data)
>>> n_scan, n_mz = im.get_size()
>>>
>>> for ii in range(n_mz):
...     ic = im.get_ic_at_index(ii)
...     ic_smooth = savitzky_golay(ic)
...     ic_bc = tophat(ic_smooth, struct="1.5m")
...     im.set_ic_at_index(ii, ic_bc)
...
\end{verbatim}

\noindent
Now the Biller and Biemann based technique can be applied to detect peaks.

\begin{verbatim}
>>> from pyms.Deconvolution.BillerBiemann.Function import BillerBiemann
>>> peak_list = BillerBiemann(im)
>>> print len(peak_list)
9845
\end{verbatim}

\noindent
Note that this is nearly as many peaks as there are scans in the data (9865
scans). This is due to noise and the simplicity of the technique.

The number of detected peaks can be constrained by the selection of better
parameters. Parameters can be determined by counting the number of points
across a peak, and examining where peaks are found. For example, the peak list
can be found with the parameters of a window of 9 points and by combining 2
neighbouring scans if they apex next to each other;

\begin{verbatim}
>>> peak_list = BillerBiemann(im, points=9, scans=2)
>>> print len(peak_list)
3698
\end{verbatim}

The number of detected peaks has been reduced, but there are still many more
than would be expected from the sample. Functions to filter the peak list are
covered in the next section.

\section{Filtering Peak Lists}

\noindent
[ {\em This example is in pyms-test/53} ]

There are two functions to filter the list of Peak objects. The first, {\tt
rel\_threshold()}, modifies the mass spectrum stored in each peak so any
intensity that is less than a given percentage of the maximum intensity for the
peak is removed. The second, {\tt num\_ions\_threshold()} removes any peak that
has less than a given number of ions above a given threshold. Once the peak
list has been constructed, the filters can be applied by;

\begin{verbatim}
>>> from pyms.Deconvolution.BillerBiemann.Function import \
... rel_threshold, num_ions_threshold
>>> pl = rel_threshold(peak_list, percent=2)
>>> new_peak_list = num_ions_threshold(pl, n=3, cutoff=10000)
>>> print len(new_peak_list)
146
\end{verbatim}

The number of detected peaks is now more realistic of what would be expected in
the test sample.

\section{Noise analysis for peak filtering}
\noindent
[ {\em This example is in pyms-test/54} ]

In the previous section the cutoff parameter for peak filtering was set by the
user. This can work well for individual data files, but can cause problems when
applied to large experiments with many individual data files. Where experimental 
conditions have changed slightly between experimental runs, the ion intensity over
the GC-MS run may also change. This means that an inflexible cutoff value can
work for some data files, while excluding too many, or including too many peaks in
other files.

An alternative to manually setting the value for cutoff, is to use the pyms function
{\tt window\_analyzer()}. This function examines a Total Ion Chromatogram (TIC) and 
computes a value for the median absolute deviation in troughs between peaks. This 
gives an approximate threshold value above which false peaks from noise should
be filtered out.

To compute this noise value:
\begin{verbatim}
>>> from pyms.Noise.Analysis import window_analyzer
... data is a GCMS data object
>>> tic = data.get_tic()
>>> noise_level = window_analyzer(tic)
\end{verbatim}

Now the usual peak deconvolution steps are performed, and the peak list is filtered
using this noise value as the cutoff:

\begin{verbatim}
... pl is a peak list, n is number of ions above threshold
>>> peak_list = num_ions_threshold(pl, n, noise_level)
\end{verbatim}


\section{Peak area estimation}

\noindent
[ {\em This example is in pyms-test/55} ]

The Peak object does not contain any information about the width or
area of the peak when it is created. This information can be added
after the instantiation of a Peak object. The area of the peak can be
set by the {\tt set\_area()}, or {\tt set\_ion\_areas()} method of the
peak object.


The total peak area can by obtained by the {\tt peak\_sum\_area()}
function in {\tt pyms.Peak.Function}. The function determines the
total area as the sum of the ion intensities for all masses that apex
at the given peak. To calculate the peak area of a single mass, the
intensities are added from the apex of the mass peak outwards. Edge
values are added until the following conditions are met: the added
intensity adds less than 0.5\% to the accumulated area; or the added
intensity starts increasing (i.e. when the ion is common to co-eluting
compounds). To avoid noise effects, the edge value is taken at the
midpoint of three consecutive edge values.

Given a list of peaks, areas can be determined and added as follows:
\begin{verbatim}
>>> from pyms.Peak.Function import peak_sum_area
>>> for peak in peak_list:
...     area = peak_sum_area(intensity_matrix, peak)
...     peak.set_area(area)
...
\end{verbatim}

\subsection{Individual Ion Areas}

[ {\em This example is in pyms-test/56} ]
\label{sec:individual-ion-areas}

While the previous approach uses the sum of all areas in the peak to
estimate the peak area, the user may also choose to record the area of
each individual ion in each peak.

This can be useful when the intention is to later perform quantitation
based on the area of a single characteristic ion for a particular
compound. It is also essential if using the Common Ion Algorithm for
quantitation, outlined in section \ref{sec:common-ion}.

To set the area of each ion for each peak, the following code is used:

\begin{verbatim}
>>> from pyms.Peak.Function import peak_top_ion_areas
>>> for peak in peak_list:
>>>     area_dict = peak_top_ions_areas(intensity_matrix, peak)
>>>     peak.set_ion_areas(area_dict)
...
\end{verbatim}

This will set the areas of the 5 most abundant ions in each peak. If
it is desired to record more than the top five ions, the argument {\tt
  num\_ions=x} should be supplied, where {\tt x} is the number of most
abundant ions to be recorded. e.g.
\begin{verbatim}
>>>     area_dict = peak_top_ions_areas(intensity_matrix, peak, num_ions=10)
\end{verbatim}
will record the 10 most abundant ions for each peak.

The individual ion areas can be set instead of, or in addition to the
total area for each peak.

\subsection{Reading the area of a single ion in a peak}

If the individual ion areas have been set for a peak, it is possible
to read the area of an individual ion for the peak. e.g.
\begin{verbatim}
>>> peak.get_ion_area(101)
\end{verbatim}
will return the area of the m/z value 101 for the peak. If the area of
that ion has not been set (i.e. it was not one of the most abundant
ions), the function will return {\tt None}.


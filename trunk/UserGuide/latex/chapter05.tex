% chapter05.tex

 %%%%%%%%%%%%%%%%%%%%%%%%%%%%%%%%%%%%%%%%%%%%%%%%%%%%%%%%%%%%%%%%%%%%%%%%%%%%%
 %                                                                           %
 %    PyMS documentation                                                     %
 %    Copyright (C) 2005-8 Vladimir Likic                                    %
 %                                                                           %
 %    The files in this directory provided under the Creative Commons        %
 %    Attribution-NonCommercial-NoDerivs 2.1 Australia license               %
 %    http://creativecommons.org/licenses/by-nc-nd/2.1/au/                   %
 %    See the file license.txt                                               %
 %                                                                           %
 %%%%%%%%%%%%%%%%%%%%%%%%%%%%%%%%%%%%%%%%%%%%%%%%%%%%%%%%%%%%%%%%%%%%%%%%%%%%%

\chapter{Isotopes related calculations}

\section
{Mass Isotope Distribution in $^{13}$C Experiments}

\noindent
[ \emph{This example is in pyms-test/09} ]

Method implemented is that of Nanchen et al \cite{nanchen09}. Higher mass ion
intensities (M, M + 1, M + 2, etc.) from $^{13}$C tracer experiments will
include natural abundance of non-C isotopes and also natural abundance of
$^{13}$C isotopes inside the GC-MS derivatization reagent, if applicable. The
function takes in the experimental ion intensities, which are obtained prior via
integration of individual ion chromatograms, calculates the corresponding mass
distribution vector and performs the correction  for natural isotope abundances
of C, O, N, H, Si, and S atoms, as well as any unlabelled biomass (this corrects
for the fact that in practice substrate is never 100\%\ labelled). The result is
a corrected mass distribution vector. Fractional labelling value is also
provided as a check, and it should be equal to the fractional labelling of the
input substrate.

Experimental data is entered using the 'mdv' variable. In the example of
alanine fragment (M-57)+ \cite{nanchen09} there are n = 3 exogenous (non-natural
abundance) C atoms, and the length of the mass distribution vector
is chosen to be n+1=4. Hence only first four intensities (ion counts)
from the mass spectrum, corresponding to M, M+1, M+2 and M+3, are entered.

\begin{verbatim}
>>> mdv = [737537, 179694, 88657, 178433]
>>> n = len(mdv) - 1
\end{verbatim}

Determine the number of C, O, N, H, Si, and S atoms in the fragment,
noting that the number of C atoms excludes C atoms which may contain
exogenous $^{13}$C atoms. For the alanine (M-57)+ fragment these
numbers are C=8, O=2, N=1, H=26, Si=2, and S=0.

\begin{verbatim} 
>>> atoms = { 'c':8, 'o':2, 'n':1, 'h':26, 'si':2, 's':0}
\end{verbatim}

Import the relevant modules:

\begin{verbatim}
>>> import numpy
>>> import pyms.Isotope.MFRA.Function
>>> import pyms.Isotope.MFRA.Constants
\end{verbatim}

Calculate the overall correction matrix:

\begin{verbatim}
>>> c_corr = pyms.Isotope.MFRA.Function.overall_correction_matrix(n, mdv, atoms)

Calculating c correction matrix

Calculating h correction matrix

Calculating si correction matrix

Calculating o correction matrix

Calculating n correction matrix

Calculating s correction matrix

Calculated overall correction matrix.

>>> c_corr array([[0.77152972 , 0.         , 0.         , 0.        ],
                  [0.1508547  , 0.77152972 , 0.         , 0.        ],
                  [0.06721399 , 0.1508547  , 0.77152972 , 0.        ],
                  [0.00866575 , 0.06721399 , 0.1508547  , 0.77152972]])
\end{verbatim}

Calculate the exclusive mass isotope distribution of the carbon skeleton:

\begin{verbatim}
>>> mdv_alpha_star = pyms.Isotope.MFRA.Function.c_mass_isotope_distr(mdv, c_corr)
>>> mdv_alpha_star array([[ 0.7729932 ],
                          [ 0.03719172],
                          [ 0.01830562],
                          [ 0.17150946]])
\end{verbatim}

Correct for unlabelled biomass. This example accounts for the contribution
of 1\%\ unlabelled biomass.

\begin{verbatim}
>>> f_unlablelled = 0.01 
>>> mdv_aa = pyms.Isotope.MFRA.Function.corr_unlabelled(n, mdv_alpha_star, f_unlabelled)
>>> mdv_aa array([[ 0.77102099],
                  [ 0.03725005],
                  [ 0.01848709],
                  [ 0.17324187]])
\end{verbatim}

Calculate the fractional labelling:

\begin{verbatim}
>>> fl = pyms.Isotope.MFRA.Function.fract_labelling(n, mdv_aa)
Fractional labelling FL: 0.197983278219
\end{verbatim}
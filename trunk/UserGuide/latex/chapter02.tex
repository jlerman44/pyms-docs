% chapter02.tex

 %%%%%%%%%%%%%%%%%%%%%%%%%%%%%%%%%%%%%%%%%%%%%%%%%%%%%%%%%%%%%%%%%%%%%%%%%%%%%
 %                                                                           %
 %    PyMS documentation                                                     %
 %    Copyright (C) 2005-8 Vladimir Likic                                    %
 %                                                                           %
 %    The files in this directory provided under the Creative Commons        %
 %    Attribution-NonCommercial-NoDerivs 2.1 Australia license               %
 %    http://creativecommons.org/licenses/by-nc-nd/2.1/au/                   %
 %    See the file license.txt                                               %
 %                                                                           %
 %%%%%%%%%%%%%%%%%%%%%%%%%%%%%%%%%%%%%%%%%%%%%%%%%%%%%%%%%%%%%%%%%%%%%%%%%%%%%

\chapter{Using PyMS}

\section{Reading GC-MS data}

\subsection{ANDI-MS data format}

The field of chromatography--mass spectrometry is ruled by proprietary, closed
data formats.  Anything that comes even closse to an open standard is the
incomplete and now obsolete ANDI-MS data format.

ANDI-MS data format stands for Analytical Data Interchange for Mass
Spectrometry, and was developed for the description of mass spectrometric
data developed in 1994 by Analytical Instrument Association. ANDI-MS 
is essentially a recommendation, and it is up to individual vendors of
mass spectrometry processing software to implement "export to ANDI-MS"
feature in their software. Furthermore, it is vendor's good will to
implement ANDI-MS specifications properly. Because of these limitations
it is difficult to be certain that one can properly read ANDI-MS files
from a particular vendor without testing this first.

\subsection{Reading the data in PyMS}

The PyMS package pyms.IO provides capabilities to read the raw GC-MS
data stored in the ANDI-MS format. The function IO.ANDI.ChemStation()
provides the interface to ANDI-MS data files saved from Agilent
ChemStation software. The name is a reminder that this function has
been reasonably tested only on the data exported from Agilent ChemStation.

In an interactive session from Python, the ANDI-MS file can be loaded
in the memory as follows:

\begin{verbatim}
>>> from pyms import IO
>>> data = IO.ANDI.Class.ChemStation('0510_217.CDF')
 -> Processing netCDF file '0510_217.CDF'
    [ 2784 scans, masses from 50 to 550 ]
>>>
\end{verbatim}

Where '0510\_217.CDF' is the name of the GC-MS data saved in the ANDI-MS
format from the Agilent ChemStation software.  The above command creates
the object 'data' which is an {\em instance} of the class IO.ANDI.ChemStation.
The instance 'data' has several attributes and methods associated with it:

\begin{itemize}

\item {\tt get\_filename()} -- Returns the name of the file from which
the data was loaded. Usage example:

\begin{verbatim}
>>> data.get_filename()
'0510_217.CDF'
\end{verbatim}

\item {\tt get\_ic\_at\_index(i)} -- Returns an IonChromatogram object
at index i.  For example, to get the first ion chromatogram from the data
matrix:

\begin{verbatim}
>>> ic = data.get_ic_at_index(1)
\end{verbatim}

\noindent
An IonChromatogram object is the one dimensional time vector containing
mass intensities.  One often deals with two types of IonChromatogram
objects: ion chromatograms at particular m/z value (for example, ion
chromatograms at m/z = 65), or total ion chromatograms (TICs), which
contain the sum of intensities for all masses at any given time point. 
The nature of an IonChromatogram object can be revealed by the content
of the attribute '\_mass', which is set to None if the ion chromatogram
is TIC; otherwise it contains the m/z value of the ion chromatogram.
Continuing the previous example:

\begin{verbatim}
>>> ic._mass
51
\end{verbatim}

\noindent
This shows that the first ion chromatogram in the data file is for
m/z = 51.

\item {\tt get\_ic\_at\_mass(mz)} -- Returns an IonChromatogram
object corresponding to given m/z. For example, to get the ion
chromatogram that corresponds to m/z = 73:

\begin{verbatim}
>>> ic = data.get_ic_at_mass(73)
>>> ic._mass
73
\end{verbatim}

\item {\tt get\_intensity\_matrix()} -- Returns the entire data
matrix, i.e. time vs m/z as numarray object. Usage example:

\begin{verbatim}
>>> im = data.get_intensity_matrix()
>>> len(im)
2784
>>> len(im[0])
501
\end{verbatim}

This data matrix contains 2784 time points (MS scans) and each time
point corresponds to a mass spectrum of 501 m/z points.

\end{itemize}

%%% section %%%%%%%%%%%%%%%%%%%%%%%%%%%%%%%%%%%%%%%%%%%%%%%%%%%%%%%%%%%%%%%%

\section{Minmax peak detector}

\subsection{Introduction}

Minmax peak detector is the simplest kind of a peak finding algoritam for
TIC. It operates by finding peak maxima, and then attempting to determine
peak boundaries. Cursory evidence suggests that gives results similar to
the ChemStation peak detection, but this was not examined rigorously.
The purpose of this algorithm is to provide an example of how 1D peak
pickin can be implemented in PyMS. {\em At present the Minmax algorithm
was not properly tested, do not use it for critical publication quality
results}.

\subsection{A brief description of the algorithm}

Many peak detection algorithms are used in practice to process GC/LC-MS data,
but only a few are fully documented, most notable those of open source
projects MZmine \cite{katajamaa06} and XCMS \cite{smith06}.  MZmine detects
peaks by finding local maxima of a certain width \cite{katajamaa06}. In XCMS
peaks are detected by using an empirical signal-to-noise cutoff after matched
filtration with a second-derivative Gaussian \cite{smith06}. PyMS peak
detection procedure was developed in-house, and relies on finding local
maxima and local minima in the signal, followed by a subsequent refinement
of peak left and right boundaries. Peak detection depends on two input
parameters: window width over which a peak is expected to be a global maximum,
and the scaling factor $S$ used to calculate the intensity threshold $S
\sigma$ which must be exceeded at the peak apex. The noise level $\sigma$
is estimated prior to peak detection by repeatedly calculating median
absolute deviation (MAD -- a robust estimate of the average deviation) over
randomly placed windows and taking the minimum. A detailed description of
procedures for peak detection follows. 

\begin{enumerate}
\item {\bf Extracting local maxima}. Initially, an ordered list of
local maxima in the signal with an intensity larger than a threshold
is compiled. Two input parameters are specified by the user: the
width of a window over which the peak is required to be a global
maximum ($W$); and (2) the scaling factor $S$ used to calculate intensity
threshold $S \sigma$, where $S$ is the noise level estimated previously
(defaults: $W = 2$ data points, $S = 10$).  User specified window is
centered on each point of the signal, and the point is deemed
to be a local maximum if the following is satisfied:

   \begin{enumerate}
   \item It is equal or greater than all of the points within
         the window W.
   \item It is greater than at least one point in the half-window
         interval to the left, and at least one point in the
         half-window interval to the right.\footnote{This is
         to reject points within intervals of uniform intensity}
   \item Any point closer to the edge of the signal than half-window
         is rejected.
   \end{enumerate}

Intensity at each local maxima is tested, and those that have the
intensity below the threshold N*S are rejected. Accepted local
maxima are compiled into a list.

\item {\bf Determination of peak left/right boundaries}. For each
local maxima (base maximum) the stretch of the signal between itself
and the next local maximum on either side is extracted.  These
two signal slices are searched for the first local minimum in
the direction away from the base maximum point itself. The local
maxima are defined in a very similar manner as the local maxima
in the previous step.  A point is deemed to be a local maximum if:

   \begin{enumerate}
   \item It is equal or smaller than all the points within the
        window W.
   \item It is smaller than at least one point in the half-window
        interval to the left, and at least one point in the
        half-window interval on the right.
   \item Any point closer to the edge of the slice than half-window
        is rejected. This has the effect that the boundary point
        cannot approach next peak's apex closer than half-window.
   \item If no minimum point is found, set the boundary point to
        the point furtherest away from the base maximum, but
        outside to the half-window range of the adjacent peak.
   \end{enumerate}

\item {\bf Elimination of peak overlaps}. In spectra dense with
peaks peak boundaries as found in the step (2) may overlap due to
the effect of user supplied window. The list of pre-peaks is
searched for overlapping peaks. In overlapping peaks the right
boundary of the lower retention time peak overlaps with the
left boundary of the higher retention time peak. The overlapping
boundaries are resolved by finding the point of minimum intensity
between the two peaks (the split point). The peak boundaries are
set to one point to the left from the split point for the right
boundary of the lower retention time peak, and to one point to
the right from the split point for the left boundary of the
higher retention time peak.

\item {\bf Correction for long tails}.  In this step peak boundaries
are adjusted to remove stretches of near-uniform intensities
(i.e. long tails). Each peak is divided at the apex into two
halves, and each half is processed individually in the
boundary-to-apex direction. A line is fitted through $M$ points
from the boundary in the least-squares sense. Prior to calculating
the angle between the line and the retention time axis, the
rise in intensity is normalized with the intensity at the peak
apex.  If this angle is below the user specified cutoff ($Q$)
the boundary point is dropped, and the process is repeated.
This adjustment is repeated until the best fit through $M$ points
from the boundary gives an angle greater than the cutoff. The
parameters $M$ and $Q$ are user specified (defaults: $M = 3$,
$Q = 1.0^\circ$).

\end{enumerate}


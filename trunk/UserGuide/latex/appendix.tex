% appendix.tex

 %%%%%%%%%%%%%%%%%%%%%%%%%%%%%%%%%%%%%%%%%%%%%%%%%%%%%%%%%%%%%%%%%%%%%%%%%%%%%
 %                                                                           %
 %    PyMS documentation                                                     %
 %    Copyright (C) 2005-8 Vladimir Likic                                    %
 %                                                                           %
 %    The files in this directory provided under the Creative Commons        %
 %    Attribution-NonCommercial-NoDerivs 2.1 Australia license               %
 %    http://creativecommons.org/licenses/by-nc-nd/2.1/au/                   %
 %    See the file license.txt                                               %
 %                                                                           %
 %%%%%%%%%%%%%%%%%%%%%%%%%%%%%%%%%%%%%%%%%%%%%%%%%%%%%%%%%%%%%%%%%%%%%%%%%%%%%

\appendix
\addcontentsline{toc}{chapter}{Appendix}
\chapter{Processing GC-MS data using PyMS}

This Chapter presents different test examples where PyMS is used to process GC-MS 
data from different instruments. 


\section{Processing GC-Quad data}

[ {\em This example is in pyms-test/A1} ]

\subsection{Instrument}
\noindent

\begin{itemize}
  \item Maker: Agilent
  \item Software: Chemstation
\end{itemize}

\subsection{Raw Data}
\noindent

\begin{itemize}
  \item File:  gc01\_0812\_066.cdf
  \item Location: PyMS/data/
  \item Nature: Metabolite mix
  \item Processing: This is the raw cdf file generated by the instrument and no further 
   processing was done using Chemstation. 
\end{itemize}


\subsection{Processing using PyMS}
\noindent

As detailed in the previous chapters, PyMS processes this data by binning the data, 
smoothing the data, removing the baseline, deconvolute peaks, filtering peaks, 
setting the mass range, removing uninformative ions and estimating peak areas. 

GC-Quad data is made up of approximately 3 scans per second. As a result of this, 
the values of the following parameters in PyMS are unique for this dataset. 

\begin{itemize}
  \item \emph{points} in Biller Biemann algorithm

   The parameter \emph{points} in the Biller Biemann algorithm defines the width of the window 
   that scans across the ion chromatogram in order to define a peak. The GC-Quad is made 
   up of only a few scans per second and hence the peaks form over only a few number of 
   scans resulting in a lower value defining the window width. In this example, this 
   value is set to 3. 

  \item \emph{scans} in Biller Biemann algorithm

   The parameter \emph{scans} in the Biller Biemann algorithm defines the number of consecutive 
   scans across which a peak can be defined. The near optimal value of this parameter 
   for GC-Quad data that is used in this example is 2. 

  \item \emph{threshold} in peak filtering

   In peak filtering, the parameter \emph{threshold (t)} defines the intensity above which a certain 
   number of ions needs to be represented to be accounted in and not to be filtered out. 
   The near optimal value of this parameter for GC-Quad data that is used in this example 
   is 10000. 

\end{itemize}


\section{Processing GC-TOF data}

[ {\em This example is in pyms-test/A2} ]

\subsection{Instrument}
\noindent

\begin{itemize}
  \item Maker: Leco Pegasus
  \item Software: ChromaTOF
\end{itemize}

\subsection{Raw Data}
\noindent

\begin{itemize}
  \item File: MM-10.0\_1\_no\_processing.cdf
  \item Location: PyMS/data/
  \item Nature: Metabolite mix
  \item Processing: This is the raw cdf file generated by the instrument and no further 
   processing was done using ChromaTOF. 
\end{itemize}


\subsection{Processing using PyMS}
\noindent

PyMS processes this data through the following steps; binning the data, smoothing the data, 
removing the baseline, deconvolute peaks, filtering peaks, setting the mass range, 
removing uninformative ions and estimating peak areas. 

GC-TOF data is made up of nearly 10 scans per second. As a result of this, the values of 
the following parameters in PyMS are unique for this dataset. 

\begin{itemize}
  \item \emph{points} in Biller Biemann algorithm

   The GC-TOF is made up of more scans per second and as a result the data is more dense when 
   compared to the GC-Quad data. Therefore the peaks form over more number of scans resulting 
   in a higher value defining the window width. In this example, this value is set to 15. 

  \item \emph{scans} in Biller Biemann algorithm

   Due to the dense nature of the data, a peak can have its maximum value over more than 1 scan. 
   In this example, this value is set to 3. 

  \item \emph{threshold} in peak filtering

   The near optimal value of this parameter for GC-TOF data that is used in this example is 4000. 

\end{itemize}
